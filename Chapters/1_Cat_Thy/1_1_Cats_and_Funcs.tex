\section{Important Fundamentals}

The basic idea of category theory is to reason collectively with large classes of mathematical objects. It is often useful to talk about a `set of all sets' or a `set of all groups'. Unfortunately, if we reason about these naïvely, we run the risk of running into paradoxes, such as Russell's paradox. Category theory provides a way to reason about these large classes of objects without running into these paradoxes.

In this module, we will not be too precise about what constitutes a \textit{class}; this is actually a very important choice in category theory, and the fact that we will not be precise about this makes our treatment of the subject fundamentally imprecise. Nevertheless, our treatment will be rigorous enough for the purposes of studying homological algebra.

With this disclaimer in mind, we are ready to begin.

\subsection{Categories and Functors}

\begin{boxdefinition}[Category]
    A \textbf{category}
\end{boxdefinition}

\subsection{Properties of Morphisms}